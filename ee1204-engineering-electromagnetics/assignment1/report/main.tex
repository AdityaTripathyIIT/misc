\documentclass[letterpaper, 11pt]{extarticle}
% \usepackage{fontspec}

% ==================================================

% document parameters
% \usepackage[spanish, mexico, es-lcroman]{babel}
\usepackage[english]{babel}
\usepackage[margin = 1in]{geometry}

% ==================================================

% Packages for math
\usepackage{mathrsfs}
\usepackage{amsfonts}
\usepackage{amsmath}
\usepackage{amsthm}
\usepackage{amssymb}
\usepackage{physics}
\usepackage{dsfont}
\usepackage{esint}

% ==================================================

% Packages for writing
\usepackage{enumerate}
\usepackage[shortlabels]{enumitem}
\usepackage{framed}
\usepackage{csquotes}

% ==================================================

% Miscellaneous packages
\usepackage{float}
\usepackage{tabularx}
\usepackage{xcolor}
\usepackage{multicol}
\usepackage{subcaption}
\usepackage{caption}
\captionsetup{format = hang, margin = 10pt, font = small, labelfont = bf}

% Citation
\usepackage[round, authoryear]{natbib}

% Hyperlinks setup
\usepackage{hyperref}
\definecolor{links}{rgb}{0.36,0.54,0.66}
\hypersetup{
   colorlinks = true,
    linkcolor = black,
     urlcolor = blue,
    citecolor = blue,
    filecolor = blue,
    pdfauthor = {Author},
     pdftitle = {Title},
   pdfsubject = {subject},
  pdfkeywords = {one, two},
  pdfproducer = {LaTeX},
   pdfcreator = {pdfLaTeX},
   }

\usepackage{titlesec}
\usepackage[many]{tcolorbox}

% Adjust spacing after the chapter title
\titlespacing*{\chapter}{0cm}{-2.0cm}{0.50cm}
\titlespacing*{\section}{0cm}{0.50cm}{0.25cm}

% Indent 
\setlength{\parindent}{0pt}
\setlength{\parskip}{1ex}

% --- Theorems, lemma, corollary, postulate, definition ---
% \numberwithin{equation}{section}

\newtcbtheorem[]{problem}{Problem}%
    {enhanced,
    colback = black!5, %white,
    colbacktitle = black!5,
    coltitle = black,
    boxrule = 0pt,
    frame hidden,
    borderline west = {0.5mm}{0.0mm}{black},
    fonttitle = \bfseries\sffamily,
    breakable,
    before skip = 3ex,
    after skip = 3ex
}{problem}

\tcbuselibrary{skins, breakable}

% --- You can define your own color box. Just copy the previous \newtcbtheorm definition and use the colors of yout liking and the title you want to use.


% --- Basic commands ---
%   Euler's constant
\newcommand{\eu}{\mathrm{e}}

%   Imaginary unit
\newcommand{\im}{\mathrm{i}}

%   Sexagesimal degree symbol
\newcommand{\grado}{\,^{\circ}}

% --- Comandos para álgebra lineal ---
% Matrix transpose
\newcommand{\transpose}[1]{{#1}^{\mathsf{T}}}

%%% Comandos para cálculo
%   Definite integral from -\infty to +\infty
\newcommand{\Int}{\int\limits_{-\infty}^{\infty}}

%   Indefinite integral
\newcommand{\rint}[2]{\int{#1}\dd{#2}}

%  Definite integral
\newcommand{\Rint}[4]{\int\limits_{#1}^{#2}{#3}\dd{#4}}

%   Dot product symbol (use the command \bigcdot)
\makeatletter
\newcommand*\bigcdot{\mathpalette\bigcdot@{.5}}
\newcommand*\bigcdot@[2]{\mathbin{\vcenter{\hbox{\scalebox{#2}{$\m@th#1\bullet$}}}}}
\makeatother

%   Hamiltonian
\newcommand{\Ham}{\hat{\mathcal{H}}}

%   Trace
\renewcommand{\Tr}{\mathrm{Tr}}

% Christoffel symbol of the second kind
\newcommand{\christoffelsecond}[4]{\dfrac{1}{2}g^{#3 #4}(\partial_{#1} g_{#2 #4} + \partial_{#2} g_{#1 #4} - \partial_{#4} g_{#1 #2})}

% Riemann curvature tensor
\newcommand{\riemanncurvature}[5]{\partial_{#3} \Gamma_{#4 #2}^{#1} - \partial_{#4} \Gamma_{#3 #2}^{#1} + \Gamma_{#3 #5}^{#1} \Gamma_{#4 #2}^{#5} - \Gamma_{#4 #5}^{#1} \Gamma_{#3 #2}^{#5}}

% Covariant Riemann curvature tensor
\newcommand{\covariantriemanncurvature}[5]{g_{#1 #5} R^{#5}{}_{#2 #3 #4}}

% Ricci tensor
\newcommand{\riccitensor}[5]{g_{#1 #5} R^{#5}{}_{#2 #3 #4}}


\usepackage{circuitikz}
\usepackage{float}
\usepackage{esint}
%\usepackage{pxfonts}
\usepackage{graphicx}
\newtheoremstyle{observationstyle}
  {3pt}   
  {3pt}   
  {\itshape}  
  {}  
  {\bfseries}  
  {.}  
  {.5em}  
  {}  
\theoremstyle{observationstyle}
\newtheorem{observation}{Observation}
\newcommand{\uvec}[1]{\boldsymbol{\hat{\textbf{#1}}}}
\begin{document}

\begin{Large}
    \textsf{\textbf{Assignment 1}}
\end{Large}

\vspace{1ex}

\textsf{\textbf{Student:}} \text{Aditya Tripathy}, \href{mailto:ee24btech11001@iith.ac.in}{\texttt{ee24btech11001@iith.ac.in}}\\

\vspace{2ex}


\begin{problem}{}{}

\begin{enumerate}[(a)]
    \item The fields due to the charges at a point $z \uvec{k}$ can be written as follows:
    \begin{align*}
        \vec{E_1} = \frac{1}{4\pi\epsilon_0} \frac{q\left(z\uvec{k} - \frac{d}{2}\uvec{k}\right)}{\left \Vert z\uvec{k} - \frac{d}{2} \uvec{k} \right \Vert^3} \\
         \vec{E_2} = \frac{1}{4\pi\epsilon_0} \frac{q\left(z\uvec{k} + \frac{d}{2}\uvec{k}\right)}{\left \Vert z\uvec{k} + \frac{d}{2} \uvec{k} \right \Vert^3} 
    \end{align*}
    Now using superposition of electric field vectors, net field at the point $z\uvec{k}$ is written as,
    \begin{align*}
        \vec{E} &= \vec{E_1} + \vec{E_2}\\
        \implies \vec{E} &= 
        \begin{cases}
            \frac{q\uvec{k}}{4\pi\epsilon_0} \left(\frac{1}{\left(z-\frac{d}{2}\right)^2} + \frac{1}{\left(z+\frac{d}{2}\right)^2}\right) & z >\frac{d}{2}\\
             \frac{q\uvec{k}}{4\pi\epsilon_0} \left(\frac{1}{\left(z+\frac{d}{2}\right)^2} - \frac{1}{\left(z-\frac{d}{2}\right)^2}\right) & \frac{-d}{2}<z<\frac{d}{2}\\
            \frac{-q\uvec{k}}{4\pi\epsilon_0} \left(\frac{1}{\left(z-\frac{d}{2}\right)^2} + \frac{1}{\left(z+\frac{d}{2}\right)^2}\right) & z <\frac{-d}{2}
        \end{cases}
    \end{align*}
    On simplification,
    \begin{align*}
        \vec{E} &= 
        \begin{cases}
            \frac{2q\uvec{k}}{4\pi\epsilon_0} \left(\frac{z^2 + \frac{d^2}{4}}{\left(z^2 - \frac{d^2}{4}\right)^2}\right) & z >\frac{d}{2}\\
             \frac{-2q\uvec{k}}{4\pi\epsilon_0} \left(\frac{zd}{\left(z^2 - \frac{d^2}{4}\right)^2}\right) & \frac{-d}{2}<z<\frac{d}{2}\\
            \frac{-2q\uvec{k}}{4\pi\epsilon_0} \left(\frac{z^2 + \frac{d^2}{4}}{\left(z^2 - \frac{d^2}{4}\right)^2}\right) & z <\frac{-d}{2}
        \end{cases}
    \end{align*}

    \item The fields due to the charges at a point $x \uvec{i}$ can be written as follows,
    \begin{align*}
    \vec{E_1} = \frac{1}{4\pi\epsilon_0} \frac{q\left(x\uvec{i} - \frac{d}{2}\uvec{k}\right)}{\left \Vert x\uvec{i} - \frac{d}{2} \uvec{k} \right \Vert^3} = \frac{q}{4\pi\epsilon_0} \frac{x\uvec{i} - \frac{d}{2}\uvec{k}}{\left( x^2+\frac{d^2}{4}\right)^{\frac{3}{2}}}\\
    \vec{E_2} = \frac{1}{4\pi\epsilon_0} \frac{q\left(x\uvec{i} + \frac{d}{2}\uvec{k}\right)}{\left \Vert x\uvec{i} + \frac{d}{2} \uvec{k} \right \Vert^3} = \frac{q}{4\pi\epsilon_0} \frac{x\uvec{i} + \frac{d}{2}\uvec{k}}{\left( x^2+\frac{d^2}{4}\right)^{\frac{3}{2}}}\\
    \end{align*}
    Again, using superposition, the net field at the point $x\uvec{i}$ can be written as,
    \begin{align*}
        \vec{E} &= \vec{E_1} + \vec{E_2}\\
        \implies \vec{E} &= \frac{q}{4\pi\epsilon_0} \frac{x\uvec{i} - \frac{d}{2}\uvec{k}}{\left( x^2+\frac{d^2}{4}\right)^{\frac{3}{2}}} + \frac{q}{4\pi\epsilon_0} \frac{x\uvec{i} + \frac{d}{2}\uvec{k}}{\left( x^2+\frac{d^2}{4}\right)^{\frac{3}{2}}}
        \\
        \implies \vec{E} &= \frac{2q}{4\pi\epsilon_0}\frac{x\uvec{i}}{\left( x^2+\frac{d^2}{4}\right)^{\frac{3}{2}}}
    \end{align*}

    \item If the sign of the charge at $\frac{-d}{2}\uvec{k}$ is flipped, we get
    \begin{enumerate}[(a)]
        \item 
        \begin{align*}
            \vec{E} &= \vec{E_1} - \vec{E_2} \\
            \implies \vec{E} &= 
            \begin{cases}
            \frac{q\uvec{k}}{4\pi\epsilon_0} \left(\frac{1}{\left(z-\frac{d}{2}\right)^2} - \frac{1}{\left(z+\frac{d}{2}\right)^2}\right) & z >\frac{d}{2}\\
             \frac{-q\uvec{k}}{4\pi\epsilon_0} \left(\frac{1}{\left(z+\frac{d}{2}\right)^2} + \frac{1}{\left(z-\frac{d}{2}\right)^2}\right) & \frac{-d}{2}<z<\frac{d}{2}\\
            \frac{q\uvec{k}}{4\pi\epsilon_0} \left(\frac{1}{\left(z+\frac{d}{2}\right)^2} - \frac{1}{\left(z-\frac{d}{2}\right)^2}\right) & z <\frac{-d}{2}
        \end{cases}
        \end{align*}
        On simplification,
        \begin{align}
            \vec{E} = 
                \begin{cases}
                     \frac{2q\uvec{k}}{4\pi\epsilon_0} \left(\frac{zd}{\left(z^2 - \frac{d^2}{4}\right)^2}\right)& z >\frac{d}{2}\\
                    \frac{-2q\uvec{k}}{4\pi\epsilon_0} \left(\frac{z^2 + \frac{d^2}{4}}{\left(z^2 - \frac{d^2}{4}\right)^2}\right) & \frac{-d}{2}<z <\frac{d}{2}\\
                    \frac{-2q\uvec{k}}{4\pi\epsilon_0} \left(\frac{zd}{\left(z^2 - \frac{d^2}{4}\right)^2}\right) & \frac{-d}{2}<z<\frac{d}{2}
                \end{cases}
        \end{align}

        \item 
        \begin{align*}
            \vec{E} &= \vec{E_1} - \vec{E_2} \\
            \implies \vec{E} &= \frac{-q}{4\pi\epsilon_0}\frac{d\uvec{i}}{\left( x^2+\frac{d^2}{4}\right)^{\frac{3}{2}}}
        \end{align*}
    \end{enumerate}
\end{enumerate}
\end{problem}
\begin{problem}{}{}
\begin{enumerate}[(a)]
    \item  { The situation described in the problem can be represented in the following free-body diagram,
        \begin{figure}[H]
        \centering
        \resizebox{0.5\textwidth}{!}{
            \begin{circuitikz}
                \tikzstyle{every node}=[font=\normalsize]
                \draw [->, >=Stealth] (2.75,6.75) -- (2.75,10.25);
                \draw [->, >=Stealth] (2.75,6.75) -- (9.25,6.75);
                \draw  (4.25,6.75) circle (1cm);
                \draw  (7.25,6.75) circle (1cm);
                \draw [->, >=Stealth] (4.25,7) -- (2.75,7);
                \draw [->, >=Stealth] (3.5,7) -- (5.75,7);
                \draw [ fill={rgb,255:red,3; green,3; blue,3} ] (5.25,7) circle (0cm);
                \node [font=\normalsize] at (5.75,7.5) {$F_{coulomb}$};
                \node [font=\normalsize] at (3,7.5) {$F_{sp}$};
                \node [font=\normalsize] at (4.25,6.5) {x=x};
                \node [font=\normalsize] at (7.25,6.5) {x=d};
                \node [font=\normalsize] at (9.5,6.75) {X};
                \node [font=\normalsize] at (2.75,10.5) {Y};
        \end{circuitikz}
    }
    \end{figure} 
    For the sphere at $x = x$ to be at equilibrium, 
    \begin{align*}
        \sum \vec{F_x} &= 0\\
        \implies k(x)(-\uvec{i}) &+ \frac{1}{4\pi\epsilon_0}\frac{(-Q^2)(x - d)\uvec{i}}{\left \Vert (x-d)\uvec{i}\right \Vert^3} = 0\\
        \implies \frac{1}{4\pi\epsilon_0}\frac{(Q^2)}{\left (d-x\right )^2} &= kx\\
        \implies Q &= \sqrt{4\pi k\epsilon_0x(d-x)^2}
        \end{align*}
        }
    \item
        If we observe the expression for Q, for $x \in [0, d]$, the maximum value of Q is achieved when,
            \begin{align*}
                \frac{d}{dx}\left( x(d-x)^2\right) &= 0\\
                \implies (d-x)^2 -2x(d-x) &= 0\\
                \implies d^2 + x^2 - 2xd -2xd + 2x^2 &= 0\\
                \implies 3x^2 + d^2 - 4xd &= 0\\
                \implies x &= \frac{d}{3}, d\\
            \end{align*}
        Since both the charges cannot be at $x = d$, because of infinite coulombic force but finite spring force. Therefore the maximum charge that can be measured using the given method is therefore, 
        \begin{align*}
            Q_{max} = \sqrt{\frac{16}{27}\pi k\epsilon_0d^3}
        \end{align*}
        If the charge is any larger than $Q_{max}$ then the system is never at equilibrium and the charged sphere initially at origin  will collapse towards the fixed sphere.
\end{enumerate}       
    
\end{problem}

\begin{problem}{}{}
    \begin{enumerate}[(a)]
        \item 
            \begin{align*}
                \text{Flux} &= \iint_{\mathbb{R}} (\vec{F_1}\cdot\uvec{n}) dS\\
                &=\int_{\phi=0}^{\phi=\frac{\pi}{2}}\int_{\theta = 0}^{\theta =2\pi} \left(\left(5\left\Vert\vec{a_z}\right\Vert\uvec{k}\right)\cdot{\boldsymbol{\hat{\rho}}}\right)\rho \sin(\phi)d\theta d\phi
            \end{align*}
            We know,
            \begin{align*}
                \uvec{k} = \cos(\phi)\boldsymbol{\hat{\rho}} - \sin(\phi)\boldsymbol{\hat{\phi}}
            \end{align*}
            Substituting, 
            \begin{align*}
                \text{Flux}&=\int_{\phi=0}^{\phi=\frac{\pi}{2}}\int_{\theta = 0}^{\theta =2\pi} \left(\left(5\left\Vert\vec{a_z}\right\Vert\left(\cos(\phi)\boldsymbol{\hat{\rho}} - \sin(\phi)\boldsymbol{\hat{\phi}}\right)\right)\cdot{\boldsymbol{\hat{\rho}}}\right)\rho^2 \sin(\phi)d\theta d\phi\\
                &=\int_{\phi=0}^{\phi=\frac{\pi}{2}}\int_{\theta = 0}^{\theta =2\pi} 5\left\Vert\vec{a_z}\right\Vert\rho^2 \sin(\phi)\cos \left(\phi \right)d\theta d\phi\\
                &=\int_{\phi=0}^{\phi=\frac{\pi}{2}}\int_{\theta = 0}^{\theta =2\pi} 5\left\Vert\vec{a_z}\right\Vert\rho^2 \sin(\phi)\cos \left(\phi \right)d\theta d\phi\\
                &=\int_{\phi=0}^{\phi=\frac{\pi}{2}} 10\pi\left\Vert\vec{a_z}\right\Vert\rho^2 \sin(\phi)\cos \left(\phi \right)d\theta d\phi\\
                &=5\pi\left\Vert\vec{a_z}\right\Vert\rho^2
            \end{align*}
            A simple observation would have saved us a lot of trouble,
                \begin{observation}
                    Flux of a divergence-free field through a closed surface is zero.
                \end{observation}
            If we consider the hemispherical surface along with the circular base(surface normal in the $-\uvec{k}$ direction), the flux through it will be zero, therefore
            \begin{align*}
                \Phi_1 + \Phi_2 &= 0\\
                \implies \Phi_1 &= -\Phi_2\\
                \implies \Phi_1 = -\left(\pi\rho^2\left(\uvec{-k}\right)\right)\cdot\left(5\left\Vert a_z\right\Vert\uvec{k}\right) &= 5\pi\left\Vert\vec{a_z}\right\Vert\rho^2
            \end{align*}
        \item 
            \begin{align*}
                \text{Flux} &= \iint_{\mathbb{R}_1} (\vec{F_1}\cdot\uvec{n}) dS + \iint_{\mathbb{R}_2} (\vec{F_1}\cdot\uvec{n}) dS\\
            \end{align*}
            where $\mathbb{R}_1$ is the hemispherical surface and $\mathbb{R}_2$ is the circular base.
            \begin{align*}
              \iint_{\mathbb{R}_2} \left(\vec{F_1}\cdot\uvec{n}\right) dS &= \iint_{\mathbb{R}_2} \left(\left(5z\left\Vert \vec{a_z} \right\Vert\uvec{k}\right)\cdot\left(\uvec{-k}\right)\right) dS\\
              &= \iint_{\mathbb{R}_2} \left(\left(5\left(0\right)\left\Vert \vec{a_z} \right\Vert\uvec{k}\right)\cdot\left(\uvec{-k}\right)\right) dS\\
              &= 0\\
              \iint_{\mathbb{R}_1} \left(\vec{F_1}\cdot\uvec{n}\right) dS &= \iint_{\mathbb{R}} (\vec{F_1}\cdot\uvec{n}) dS\\
                &=\int_{\phi=0}^{\phi=\frac{\pi}{2}}\int_{\theta = 0}^{\theta =2\pi} \left(\left(5z\left\Vert\vec{a_z}\right\Vert\uvec{k}\right)\cdot{\boldsymbol{\hat{\rho}}}\right)\rho \sin(\phi)d\theta d\phi
            \end{align*}
            Using, 
            \begin{align*}
                \uvec{k} &= \cos(\phi)\boldsymbol{\hat{\rho}} - \sin(\phi)\boldsymbol{\hat{\phi}}\\
                z &= \rho \sin\left(\phi\right) 
            \end{align*}
            We get,
            \begin{align*}
                \iint_{\mathbb{R}_1} \left(\vec{F_1}\cdot\uvec{n}\right) dS&=\int_{\phi=0}^{\phi=\frac{\pi}{2}}\int_{\theta = 0}^{\theta =2\pi} \left(\left(5\rho\sin\left(\phi\right)\left\Vert\vec{a_z}\right\Vert\left(\cos(\phi)\boldsymbol{\hat{\rho}} - \sin(\phi)\boldsymbol{\hat{\phi}}\right)\right)\cdot{\boldsymbol{\hat{\rho}}}\right)\rho^2 \sin(\phi)d\theta d\phi\\
                &=\int_{\phi=0}^{\phi=\frac{\pi}{2}}\int_{\theta = 0}^{\theta =2\pi} 5\rho\sin\left(\phi\right)\left\Vert\vec{a_z}\right\Vert\rho \sin(\phi)\cos \left(\phi \right)d\theta d\phi\\
                &=\int_{\phi=0}^{\phi=\frac{\pi}{2}}\int_{\theta = 0}^{\theta =2\pi} 5\left\Vert\vec{a_z}\right\Vert\rho^3 \sin^2(\phi)\cos \left(\phi \right)d\theta d\phi\\
                &=\int_{\phi=0}^{\phi=\frac{\pi}{2}} 10\pi\left\Vert\vec{a_z}\right\Vert\rho^3 \sin^2(\phi)\cos \left(\phi \right)d\theta d\phi\\
                &=\frac{10}{3}\pi\left\Vert\vec{a_z}\right\Vert\rho^3
            \end{align*}
            Therefore, total flux is given by,
            \begin{align*}
                \text{Flux} = \frac{10}{3}\pi\left\Vert\vec{a_z}\right\Vert\rho^3
            \end{align*}
        \item We can solve part (b) using Divergence Theorem
        \begin{align*}
            \mathbb{\nabla}\cdot\vec{F_1} &= \frac{\partial F_{1x}}{\partial x} + \frac{\partial F_{1y}}{\partial y} + \frac{\partial F_{1z}}{\partial z}\\
            &= \frac{\partial F_{1z}}{\partial z}\\
            &= 5\left\Vert\vec{a_z}\right\Vert
        \end{align*}
        Applying Divergence Theorem, 
        \begin{align*}
            \text{Flux} = \iint_{\mathbb{R}} \left(\vec{F_1}\cdot\uvec{n}\right)\,dS = \iiint_{\mathbb{V}}
            \mathbb{\nabla}\cdot\vec{F_1}\,dV &= 5\left\Vert\vec{a_z}\right\Vert \iiint_{\mathbb{V}}\, dV\\
            &= 5\left\Vert\vec{a_z}\right\Vert\left(\frac{2\pi\rho^3}{3}\right)\\
            &= \left\Vert\vec{a_z}\right\Vert\left(\frac{10\pi\rho^3}{3}\right)
        \end{align*}
    \end{enumerate}
\end{problem}

\begin{problem}{}{}

        Owing to the cylindrically symmetric nature of the charge distribution, the electric field must be radially emerging outwards. Exploiting this symmetry, we can use Gauss' theorem, 
        \begin{align*}
            \oiint \vec{E}\cdot d\vec{s} = \frac{Q_{enc}}{\epsilon_0}
        \end{align*}
        Taking the Gaussian surface to be a cylinder with radius $0<r<b$ and length $l$ (going into the page in the following figure), 
        \begin{figure}[!ht]
            \centering
            \resizebox{1\textwidth}{!}{%
            \begin{circuitikz}
            \tikzstyle{every node}=[font=\LARGE]
            \draw  (5.5,14.25) circle (8cm);
            \draw [ dashed] (5.5,14) circle (5.25cm);
            \draw [->, >=Stealth] (5.5,13.5) -- (9.25,17.75);
            \draw [->, >=Stealth] (5.5,13.5) -- (-1,19);
            \node [font=\LARGE] at (7,16.25) {r};
            \node [font=\Large] at (2.75,16.25) {b};
            \end{circuitikz}
            }%
            \caption{Top view of the cylindrical dielectric}
        \end{figure}
        $Q_{enc}$ can be calculated as,
        \begin{align*}
            Q_{enc} &= \int_{z=0}^{z=l}\int_{\theta = 0}^{\theta =2\pi}\int_{\rho=0}^{\rho=r}\rho_v \rho \,d\rho \,dz \, d\theta\\
            &= \int_{z=0}^{z=l}\int_{\theta = 0}^{\theta =2\pi}\int_{\rho=0}^{\rho=r}a \rho^3 \,d\rho \,dz \, d\theta\\
            &= \frac{\pi lr^4}{2}
        \end{align*}
        Now, using this in Gauss' Law,
        \begin{align*}
            \left\Vert \vec{E} \right\Vert \cdot{\left(2\pi rl\right)} &= \frac{\pi lr^4}{2\epsilon_0}\\
            \implies \vec{E} &= \frac{r^3}{4\epsilon_0}\boldsymbol{\hat{\rho}},\quad 0< r < a
        \end{align*}

        For $r \ge a$, 
        \begin{align*}
            Q_{enc} = \frac{\pi la^4}{2}
        \end{align*}
        Again using Gauss' Law, 
        \begin{align*}
            \left\Vert \vec{E} \right\Vert \cdot{\left(2\pi rl\right)} &= \frac{\pi la^4}{2\epsilon_0}\\
            \implies \vec{E} &= \frac{a^4}{4r\epsilon_0}\boldsymbol{\hat{\rho}}
        \end{align*}        
\end{problem}

\begin{problem}{}{}
    \begin{enumerate}[(a)]
        \item 
        \begin{align*}
            \left\Vert \vec{F} \right\Vert &= \sqrt{\left\Vert \vec{F_s} \right\Vert^2+\left\Vert \vec{F_{\phi}} \right\Vert^2 + \left\Vert \vec{F_z} \right\Vert^2}\\
            &= \sqrt{\left(4+(3\left(\cos\phi + \sin\phi\right)\right)^2 + 9\left(\cos\phi - \sin\phi\right)^2 + 4}\\
            &= \sqrt{\left(16+9\left(1+2\sin\phi\cos\phi\right) + 12\left(\cos\phi+\sin\phi\right)\right) + 9\left(1 - 2\cos\phi\sin\phi\right) + 4}\\
            &= \sqrt{\left(25 + 18\sin\phi\cos\phi\right) + 12\left(\cos\phi+\sin\phi\right) + 9 - 18\cos\phi\sin\phi + 4}\\
            &= \sqrt{38 + 12\left(\cos\phi+\sin\phi\right)}\\
        \end{align*}
        \begin{figure}
            \centering
            \includegraphics[width=0.3\linewidth]{5a.png}
            \caption{Field with $s=3$ as a function of $\phi$}
            \label{fig:enter-label}
        \end{figure}
    \item 
    \begin{align*}
        \left\Vert \vec{F} \right\Vert &= \sqrt{\left\Vert \vec{F_s} \right\Vert^2+\left\Vert \vec{F_{\phi}} \right\Vert^2 + \left\Vert \vec{F_z} \right\Vert^2}\\
        &= \sqrt{\left(\frac{40}{s^2+1} + 3\sqrt{2}\right)^2 + 4}
    \end{align*}
    \begin{figure}
            \centering
            \includegraphics[width=0.5\linewidth]{5b.png}
            \caption{Caption}
            \label{fig:enter-label}
        \end{figure}
    \item Divergence in cylindrical coordinates is given by,
    \begin{align*}
        \mathbf{\nabla}\cdot \vec{F} &= \frac{1}{s}\frac{\partial \left(s F_s\right)}{\partial s} + \frac{1}{s}\frac{\partial A_{\phi}}{\partial \phi} + \frac{\partial A_z}{\partial z}\\
        \frac{\partial \left(sF_s\right)}{\partial s} &= \frac{d}{ds}\left(\frac{40s}{s^2+1} + 3s\left(\cos\phi+\sin\phi\right)\right)\\
        &=\frac{40\left(s^2+1 - 2s^2\right)}{\left(s^2+1\right)^2} + 3\left(\cos\phi+\sin\phi\right)\\
        &=\frac{40\left(1 - s^2\right)}{\left(s^2+1\right)^2} + 3\left(\cos\phi+\sin\phi\right)\\
       \frac{1}{s}\frac{\partial \left(s F_s\right)}{\partial s} &=  \frac{40\left(1 - s^2\right)}{s\left(s^2+1\right)^2} + \frac{3\left(\cos\phi+\sin\phi\right)}{s}\\
       \frac{\partial A_{\phi}}{\partial \phi} &= -3\left(\sin\phi+\cos\phi\right)\\
       \frac{1}{s}\frac{\partial A_{\phi}}{\partial \phi} &= \frac{-3\left(\sin\phi+\cos\phi\right)}{s}\\
       \frac{\partial A_z}{\partial z} &= 0
    \end{align*}
    Finally, the divergence is given by,
    \begin{align*}
        \mathbf{\nabla}\cdot \vec{F} &= \frac{40\left(1 - s^2\right)}{s\left(s^2+1\right)^2}\\
    \end{align*}
    \item
    The curl of a vector field \(\mathbf{F} = F_s \hat{s} + F_\phi \hat{\phi} + F_z \hat{z} \) in cylindrical coordinates \((s, \phi, z)\) is given by:

\begin{align*}
\nabla \times \mathbf{F} =
\begin{vmatrix}
\hat{s} & \hat{\phi} & \hat{z} \\
\frac{\partial}{\partial s} & \frac{1}{s} \frac{\partial}{\partial \phi} & \frac{\partial}{\partial z} \\
F_s & F_\phi & F_z
\end{vmatrix}
\end{align*}

Expanding this determinant, the components of the curl are:

\begin{align*}
(\nabla \times \mathbf{F})_s = \frac{1}{s} \left( \frac{\partial F_z}{\partial \phi} - \frac{\partial}{\partial z} (s F_\phi) \right)
\end{align*}

\begin{align*}
(\nabla \times \mathbf{F})_\phi = \frac{\partial F_s}{\partial z} - \frac{\partial F_z}{\partial s}
\end{align*}

\begin{align*}
(\nabla \times \mathbf{F})_z = \frac{1}{s} \left[ \frac{\partial}{\partial s} (s F_\phi) - \frac{\partial F_s}{\partial \phi} \right]
\end{align*}




 \( F_z = -2z \) $\implies$ \( \frac{\partial F_z}{\partial \phi} = 0 \).
 \( F_\phi = 3(\cos\phi - \sin\phi) \), so:

  \[
  \frac{\partial F_\phi}{\partial z} = 0, \quad s F_\phi = 3s(\cos\phi - \sin\phi)
  \]

  \[
  \frac{\partial}{\partial z} (s F_\phi) = 0
  \]

Thus, \((\nabla \times \mathbf{F})_s = 0\).


 \( F_s = \frac{40}{s^2+1} + 3(\cos\phi + \sin\phi) \), so:

  \[
  \frac{\partial F_s}{\partial z} = 0
  \]

 \( \frac{\partial F_z}{\partial s} = \frac{\partial (-2z)}{\partial s} = 0 \)

Thus, \((\nabla \times \mathbf{F})_\phi = 0\).


 We already have \( s F_\phi = 3s(\cos\phi - \sin\phi) \), so:

  \[
  \frac{\partial}{\partial s} (s F_\phi) = \frac{\partial}{\partial s} [3s(\cos\phi - \sin\phi)] = 3(\cos\phi - \sin\phi)
  \]

 For \( \frac{\partial F_s}{\partial \phi} \):

  \[
  \frac{\partial F_s}{\partial \phi} = \frac{\partial}{\partial \phi} \left[ \frac{40}{s^2+1} + 3(\cos\phi + \sin\phi) \right]
  \]

  \[
  = 3(-\sin\phi + \cos\phi)
  \]

Thus,

\[
(\nabla \times \mathbf{F})_z = \frac{1}{s} \left[ 3(\cos\phi - \sin\phi) - 3(-\sin\phi + \cos\phi) \right] = 0.
\]
Since $\mathbb{\nabla}\cdot\vec{F} = 0$ field is conservative.
    \end{enumerate}
\end{problem}

\begin{problem}{}{}
    
\end{problem}

\begin{problem}{}{}
    \begin{enumerate}[(a)]
        \item From Poisson's Equation we know, 
        \begin{align*}
            \mathbb{\nabla}^2V = -\frac{\rho}{\epsilon_0}
        \end{align*}
        Writing the Laplacian in spherical coordinates, 
        \begin{align*}
\nabla^2 f = \frac{1}{r^2} \frac{\partial}{\partial r} \left( r^2 \frac{\partial f}{\partial r} \right)
+ \frac{1}{r^2 \sin\theta} \frac{\partial}{\partial \theta} \left( \sin\theta \frac{\partial f}{\partial \theta} \right)
+ \frac{1}{r^2 \sin^2\theta} \frac{\partial^2 f}{\partial \phi^2}.
        \end{align*}
        We can see that
        \begin{align*}
            \frac{\partial V}{\partial \theta} = \frac{\partial V}{\partial \phi} = 0
        \end{align*}
        And, 
        \begin{align*}
            \frac{\partial V}{\partial r} &= -\frac{V_0}{a}e^{\frac{-r}{a}}\\
            r^2\frac{\partial V}{\partial r} &= -\frac{V_0r^2}{a}e^{\frac{-r}{a}}\\
            \frac{\partial}{\partial r}\left(r^2\frac{\partial V}{\partial r}\right)&= \frac{-V_0}{a}e^{\frac{-r}{a}}\left(2r -\frac{r^3}{a}\right)\\
            \frac{1}{r^2}\frac{\partial}{\partial r}\left(r^2\frac{\partial V}{\partial r}\right)&= \frac{-V_0}{a}e^{\frac{-r}{a}}\left(\frac{2}{r} -\frac{r}{a}\right)\\
        \end{align*}
        Therefore,
        \begin{align*}
            \rho_v(r) &=  \frac{V_0\epsilon_0}{a}e^{\frac{-r}{a}}\left(\frac{2}{r} -\frac{r}{a}\right)\\
            \rho_v(a) &=  \frac{V_0\epsilon_0}{a}e^{-1}\left(\frac{2}{a} -1 \right)\\
        \end{align*}
        \item We know, 
        \begin{align*}
            \vec{E} &= -\mathbb{\nabla}V\\
            &= -\left( \frac{\partial V}{\partial r} \hat{r} + \frac{1}{r} \frac{\partial V}{\partial \theta} \hat{\theta} + \frac{1}{r \sin\theta} \frac{\partial V}{\partial \phi} \hat{\phi}\right)\\
            &= \frac{V_0}{a}e^{\frac{-r}{a}} \hat{r}
        \end{align*}
        \item To calculate the total charge, 
        \begin{align*}
            Q_{tot} &= \epsilon_0 \lim_{r \rightarrow \infty} \left(4\pi r^2\right)\left(\frac{V_0}{a}e^{\frac{-r}{a}}\right)\\
            &=0
        \end{align*}
    \end{enumerate}
\end{problem}
\begin{problem}{}{}    
    Due to the infinite size of the parallel plates in the x-y plane, the electric potential and electric field will vary only with $z$ as all points in the x-y plane are identical in terms of field and potential. 
    \begin{enumerate}[(a)]
        \item 
            With this information, we write Poisson's Equation,  
            \begin{align*}
                \mathbb{\nabla}^2V &= \frac{-\rho}{\epsilon_0}\\
                \implies \frac{\partial^2 V}{\partial x^2}+\frac{\partial^2 V}{\partial y^2}+\frac{\partial^2 V}{\partial z^2} &= \frac{-\rho}{\epsilon_0}\\
            \end{align*}
            With, 
            \begin{align*}
                \frac{\partial^2 V}{{\partial x}^2} &= 0\\              
                \frac{\partial^2 V}{{\partial y}^2} &= 0\\              
            \end{align*}
            Now, 
            \begin{align*}
                \frac{\partial^2 V}{{\partial z}^2} &= \frac{-\rho_0}{\epsilon_0}\\              
                \implies V(x, y, z) &= \frac{-\rho_0 z^2}{2\epsilon_0} + C_1z + C_2\\
            \end{align*}\
            Using the boundary conditions, 
            \begin{align*}
                V(x, y, 0) &= 0\\
                V(x, y, d) &= 0\\
                \\
                \implies C_2 &= 0\\
                \text{and} \quad C_1 &= \frac{\rho_0 d}{2\epsilon_0} 
            \end{align*}
            Finally, 
            \begin{align*}
                V(x,y,z) = \frac{-\rho_0 z^2}{2\epsilon_0} + \frac{\rho_0 dz}{2\epsilon_0}  \\
            \end{align*}
        \item 
        We know, 
        \begin{align*}
            \vec{E} &= -\mathbb{\nabla}V = -\frac{\partial V}{\partial z}\uvec{k}\\
                &= \uvec{k}\frac{\partial}{\partial z}\left(\frac{\rho_0 z^2}{2\epsilon_0} - \frac{\rho_0 dz}{2\epsilon_0}\right)\\
                &= \frac{\rho_0}{\epsilon_0}\left(z - \frac{d}{2}\right)\uvec{k}
        \end{align*}
        \item 
            \begin{enumerate}
                \item In this problem we just change one of the boundary condition, namely,   
                    \begin{align*}
                        V(x, y, d) = V_0\\
                    \end{align*}
                    So we will still get \begin{align*}
                        C_2 = 0
                    \end{align*} due to the previous boundary condition. Using the new boundary condition, we get
                    \begin{align*}
                        V_0  &= \frac{-\rho_0 d^2}{2\epsilon_0} + C_1d\\
                        \implies C_1 &= \frac{1}{d}\left(V_0 + \frac{\rho_0 d^2}{2\epsilon_0}\right)
                    \end{align*}
                    So the new expression for the potential is,
                    \begin{align*}
                        V(x, y, z) &= \frac{-\rho_0 z^2}{2\epsilon_0} + \frac{z}{d}\left(V_0 + \frac{\rho_0 d^2}{2\epsilon_0}\right)
                    \end{align*}
                \item  
                    Again
                        \begin{align*}
                            \vec{E} &= -\mathbb{\nabla}V = -\frac{\partial V}{\partial z}\uvec{k}\\
                                &= \uvec{k}\frac{\partial}{\partial z}\left(\frac{-\rho_0 z^2}{2\epsilon_0} + \frac{z}{d}\left(V_0 + \frac{\rho_0 d^2}{2\epsilon_0}\right)\right)\\
                                &= \left(\frac{-\rho_0 z}{\epsilon_0} + \frac{1}{d}\left(V_0 + \frac{\rho_0 d^2}{2\epsilon_0}\right)\right)\uvec{k}
                        \end{align*}
            \end{enumerate}
    \end{enumerate}
 \end{problem}
% =================================================
% \newpage

% \vfill

\end{document}
