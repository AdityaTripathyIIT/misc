\documentclass{article}
\usepackage{amsmath}
\usepackage{graphicx} % Required for inserting images

\title{Assignment 1}
\author{Aditya Tripathy\\EE24BTECH11001\\GROUP 9}

\begin{document}

\maketitle
\section{Question}
Obtain the $1$'s and $2$'s complements of the following binary numbers:
\begin{enumerate}
    \item[(a)] \(11100010\)
    \item[(b)] \(00011000\)
    \item[(c)] \(10111101\)
    \item[(d)] \(10100101\)
    \item[(e)] \(11000011\)
    \item[(f)] \(01011000\)
\end{enumerate}
\section{Solution}

To take the 1's complement of a binary number, we just to swap 0's and 1's.\\
For 2' complement, we add 1 to 1's complement of the binary number.

\begin{enumerate}
    \item \(11100010\)
        \begin{itemize}
            \item 1’s complement: \(00011101\)
            \item Add 1:
                  \[
                  \begin{array}{r}
                    \quad \text{\small 1} \phantom{0}\\
                    \phantom{+}00011101 \\
                    +\quad 1 \\
                    \hline
                    00011110 \\
                  \end{array}
                  \]
            \item 2’s complement: \(00011110\)
        \end{itemize}

    \item \(00011000\)
        \begin{itemize}
            \item 1’s complement: \(11100111\)
            \item Add 1:
                  \[
                  \begin{array}{r}
                    \quad \text{\small 111}\phantom{0} \\
                    \phantom{+}11100111 \\
                    +\quad 1 \\
                    \hline
                    11101000 \\
                  \end{array}
                  \]
            \item 2’s complement: \(11101000\)
        \end{itemize}

    \item \(10111101\)
        \begin{itemize}
            \item 1’s complement: \(01000010\)
            \item Add 1:
                  \[
                  \begin{array}{r}
                    \phantom{+}01000010 \\
                    +\quad 1 \\
                    \hline
                    01000011 \\
                  \end{array}
                  \]
            \item 2’s complement: \(01000011\)
        \end{itemize}

    \item \(10100101\)
        \begin{itemize}
            \item 1’s complement: \(01011010\)
            \item Add 1:
                  \[
                  \begin{array}{r}
                    \phantom{+}01011010 \\
                    +\quad 1 \\
                    \hline
                    01011011 \\
                  \end{array}
                  \]
            \item 2’s complement: \(01011011\)
        \end{itemize}

    \item \(11000011\)
        \begin{itemize}
            \item 1’s complement: \(00111100\)
            \item Add 1:
                  \[
                  \begin{array}{r}
                    \phantom{+}00111100 \\
                    +\quad 1 \\
                    \hline
                    00111101 \\
                  \end{array}
                  \]
            \item 2’s complement: \(00111101\)
        \end{itemize}

    \item[(f)] \(01011000\)
        \begin{itemize}
            \item 1’s complement: \(10100111\)
            \item Add 1:
                  \[
                  \begin{array}{r}

                    \quad \text{\small 111}\phantom{0}\\
                    \phantom{+}10100111 \\
                    +\quad 1 \\
                    \hline
                    10101000 \\
                  \end{array}
                  \]
            \item 2’s complement: \(10101000\)
        \end{itemize}
\end{enumerate}
\textbf{Question 2}\newline
Determine the base of the numbers in each case for the following operations to be correct:
\begin{enumerate}
  \item $\frac{67}{5} = 11$ \\
  \item $15 \times 3 = 51$\\
  \item $123 + 120 = 303$
\end{enumerate}
\textbf{Solution:}\newline
\begin{enumerate}
  \item Let the base of the numbers be $k$. In base 10, 
    \begin{align*}
      67 &= 6 \times k^1 + 7 \times k^0 \\
      11 &= 1 \times k^1 + 1\times k^0\\
      5 &= 5 \times k^0
    \end{align*}
  The equation in base 10 will be
    \begin{align*}
	    \frac{6k + 7}{5} &= k+1\\
	    6k+7 &= 5k + 5\\
	    6k-5k &= 5-7\\ 
	    k = -2
    \end{align*}
 The base comes out to be -2 which is not possible. Hence there is no base for which the given equation makes sense. 
  \item Let the base of the numbers be $k$
    \begin{align*}
      15 &= 1 \times k^1 + 5 \times k^0 \\
      51 &= 5 \times k^1 + 1\times k^0\\
      3 &= 3 \times k^0
    \end{align*}
  The equation in base 10 will be
    \begin{align*}
	    3(k + 5) &= 5k + 1\\
	    3k+15 &= 5k+1\\
	    2k &= 14\\
	    k &= 7
    \end{align*}
    Hence the equation holds true in base 7.
  \item Let the base of the numbers be $k$. The value of the number in base 10 will be
    \begin{align*}
      123 &= 1 \times k^2 + 2 \times k^1 + 3\times k^0 \\
      120 &= 1 \times k^2 + 2 \times k^1 + 0\times k^0\\
      303 &= 3 \times k^2 + 0 \times k^1 + 3\times k^0
    \end{align*}
  Now, 
    \begin{align*}
      (k^2 + 2k +3) + (k^2 + 2k) &= 3k^2 + 3\\
	    2k^2+4k+3 &= 3k^2+3\\
	    k^2-4k &= 0\\
	    k = 0 \\ k = 4
    \end{align*}
    Base 0 is meaningless for the given problem, hence the base in which the given equation holds is base 4.

\end{enumerate}

\textbf{Question 3}\newline
The solutions to the quadratic equation $x^2-13x+22 = 0$ are $x = 7$ and $x = 2$. What is the base of the numbers?
\newline
\textbf{Solution}\newline
Let the numbers be in base $k$ then, in base 10 the equation becomes
\begin{align*}
	13 = 1\times k^1+3\times k^0\\
	22 = 2\times k^1 + 2\times k^0\\
	x^2 - (k+3)x + (2k+2) = 0\\
\end{align*}
Since 2 and 7 are given to be solutions in the problem statement,
\begin{align*}
	49 - 7(k+3) + (2k+2) &= 0\\
	k &= 6\\
\end{align*}
\begin{align*}
	4 - 2(k+3) + (2k+2) &= 0\\
	0 &= 0 \\
\end{align*}
Arbitrary values of $k$ can be used for  $x=2$ to be a solution, however $k = 6$ is the only value for $x=7$ to be the solution. Hence the value of $k$ is 6.

\textbf{Question 4}\newline
How many printing characters are there in ASCII? How many of them are special characters (not letters or numerals)?\newline
\textbf{Solution}\newline
There are 95 printing characters in ASCII out of which 33 are special characters

\textbf{Question 5}\newline
What bit must be complemented to change an ASCII letter from capital to lowercase and vice versa?\newline
\textbf{Solution}\newline
Capital alphabets start from index 65 while lowercase alphabets start from index 97. Since the difference between the two is $32 = 2^5$ the 6th bit (corresponding to $2^5$) must be complemented to change an ASCII letter from capital to lowercase and vice versa.
\end{document}
