% Chapter Template

\chapter{Convergence - Uniform, Absolute and Pointwise}

\label{Chapter4}

\lhead{Chapter 4. \emph{Convergence of Current Fourier Series}} 


\section{What is Convergence?}
Convergence refers to whether a sequence of approximations gets closer to a target value. For a Fourier series, convergence means that the series representation of a function behaves like the function as we add more terms. That is, the partial sum:
\begin{equation}
S_N(x) = a_0 + \sum_{n=1}^{N} (a_n \cos(nx) + b_n \sin(nx))
\end{equation}
should get closer and closer to $f(x)$ as $N \to \infty$.


\section{Understanding Convergence in Fourier Series}

When we approximate a function $f(x)$ using its Fourier series, we want to know how well the series actually represents $f(x)$. The term *"convergence"* describes whether and how the sum of the Fourier series approaches the function. There are three main types of convergence we study:\\\\
1. *Pointwise Convergence* – Does the Fourier series converge to $f(x)$ at each point?\\
2. *Uniform Convergence* – Does the convergence happen evenly across the entire interval?\\
3. *Absolute Convergence* – Does the sum of the absolute values of the terms converge?\\

Each of these has different implications for how well the Fourier series represents $f(x)$, and understanding them helps in analyzing how Fourier series behave in practical applications.

\section{Pointwise Convergence}
\subsection{Definition}
A sequence of functions $S_N(x)$ converges \textit{pointwise} to a function $f(x)$ if for each fixed $x$, the sequence $S_N(x)$ gets arbitrarily close to $f(x)$ as $N \to \infty$, i.e.,
\[
\lim_{N \to \infty} S_N(x) = f(x).
\]
For a Fourier series, this means that at each specific point $x$, the sum of the first $N$ terms of the series gets closer and closer to $f(x)$ as $N$ increases.

\subsection{Mathematical Proof of Pointwise Convergence for Fourier Series}
Consider the Fourier series of a function $f(x)$:
\[
S_N(x) = \sum_{n=-N}^{N} c_n e^{j n \omega x}.
\]
We analyze whether this series converges to $f(x)$ for each $x$. The *Dirichlet conditions* guarantee pointwise convergence almost everywhere if:

1. $f(x)$ is piecewise continuous (it has a finite number of discontinuities).
2. $f(x)$ is piecewise smooth (both $f(x)$ and $f'(x)$ are piecewise continuous).

If these conditions are met, the Fourier series converges to $f(x)$ at points where $f(x)$ is continuous. However, at discontinuities, the Fourier series converges to the *midpoint* of the left-hand and right-hand limits:
\[
S_N(x) \to \frac{f(x^+) + f(x^-)}{2}.
\]
This is known as the *Gibbs phenomenon*, where the Fourier series overshoots near discontinuities.

\section{Uniform Convergence}
\subsection{Definition}
A Fourier series converges \textit{uniformly} to $f(x)$ if the error between $S_N(x)$ and $f(x)$ can be made arbitrarily small for all $x$ at the same time, i.e.,
\[
\sup_{x} |S_N(x) - f(x)| \to 0 \text{ as } N \to \infty.
\]
This means that no matter where you look, the difference between the partial sum $S_N(x)$ and $f(x)$ is consistently small across the entire interval.

\subsection{Why Fourier Series May Not Converge Uniformly}
If a Fourier series is to converge uniformly, it must satisfy the Weierstrass M-test, which roughly states that the terms of the series must decrease in a controlled way. However, if $f(x)$ has a discontinuity, uniform convergence fails due to the Gibbs phenomenon—the oscillations near the discontinuity remain significant even as $N \to \infty$.

A key theorem states that if a Fourier series converges uniformly, then it must also converge absolutely. Since Fourier series do not always converge absolutely, they often fail to be uniformly convergent.

\section{Absolute Convergence}
\subsection{Definition}
A series $\sum a_n$ is said to be \textit{absolutely convergent} if the series of absolute values $\sum |a_n|$ also converges.

For a Fourier series, absolute convergence means:
\[
\sum_{n=-\infty}^{\infty} |c_n| < \infty.
\]
This ensures that the series is well-behaved and guarantees uniform convergence.

\subsection{Why Most Fourier Series Do Not Converge Absolutely}
The Fourier coefficients $c_n$ of a function $f(x)$ typically decay as $\frac{1}{n}$ or slower. The harmonic series:
\[
\sum_{n=1}^{\infty} \frac{1}{n}
\]
diverges, which means that most Fourier series do *not* converge absolutely.



\section{Comparison of Convergence Types}
\begin{table}[h]
    \centering
    \renewcommand{\arraystretch}{1.3}
    \begin{tabular}{p{4cm} p{3cm} p{5cm} p{4cm}}
        \toprule
        \textbf{Convergence Type} & \textbf{Strength} & \textbf{Works for Discontinuous Functions?} & \textbf{Key Condition} \\
        \midrule
        Pointwise   & Weak  & Yes, but oscillates at jumps  & $S_N(x) \to f(x)$ at each point \\
        \midrule
        Uniform    & Medium  & No (fails at discontinuities) & $\sup |S_N(x) - f(x)| \to 0$ \\
        \midrule
        Absolute   & Strong  & Rarely (only for fast-decaying coefficients) & $\sum |a_n| + |b_n| < \infty$ \\
        \bottomrule
    \end{tabular}
    \caption{Comparison of convergence types in Fourier series.}
\end{table}

\section{Convergence}
Analysis of the convergence properties of the complex Fourier series representation of the current response in a series RL circuit subjected to a square wave input. We examine three types of convergence:
\begin{itemize}
  \item \textbf{Pointwise Convergence}
  \item \textbf{Uniform Convergence}
  \item \textbf{Absolute Convergence}
\end{itemize}
Each section includes step-by-step proofs with explicit integration and simplifications.

\section{Complex Fourier Series Representation of RL Circuit Response}
The RL circuit is governed by the differential equation:
\begin{equation}
L \frac{di}{dt} + R i = V(t),
\end{equation}
where $L$ is the inductance, $R$ is the resistance, and $i(t)$ is the circuit current.

For a square wave voltage input with period $T$, we expand it into its complex Fourier series:
\begin{equation}
V(t) = A\alpha + \sum_{k=-\infty}^{\infty} \frac{A}{\pi k} \sin (\pi k \alpha)e^{-j \pi k \alpha} e^{j k \omega t}, \quad \omega = \frac{2\pi}{T}.
\end{equation}
The Fourier coefficients are given by:
\begin{equation}
C_k = \frac{1}{T} \int_{0}^{T} V(t) e^{-j k \omega t} dt.
\end{equation}

Using this representation, we obtain the Fourier series for the circuit current:
\begin{equation}
i(t) = \frac{A \alpha}{R} \left(1 - e^{-\frac{R T}{L}}\right) + \sum_{k=-\infty}^{\infty} \left( \frac{A \sin (\pi k \alpha) e^{-j \pi k \alpha}}{\pi k (L j k \omega + R)} \right) \left(e^{j k \omega t} - e^{-\frac{R T}{L}}\right).
\end{equation}

\section{Pointwise Convergence}
Pointwise convergence requires that:
\begin{equation}
\lim_{N \to \infty} S_N(t) = i(t),
\end{equation}
where $S_N(t)$ is the $N$-term partial sum of the Fourier series. According to Dirichlet's theorem, a Fourier series converges pointwise to the function value wherever the function is continuous. At points of discontinuity, the Fourier series converges to the average of the left-hand and right-hand limits:
\begin{equation}
\lim_{N \to \infty} S_N(t) = \frac{V(t^+) + V(t^-)}{2}.
\end{equation}
Since the square wave has jump discontinuities, the Fourier series representation of $V(t)$ exhibits the Gibbs phenomenon.

\section{Uniform Convergence}
Uniform convergence requires:
\begin{equation}
\sup_{t \in [0,T]} |S_N(t) - i(t)| \to 0 \quad \text{as} \quad N \to \infty.
\end{equation}
The Weierstrass M-test can be used to determine uniform convergence. If there exists a sequence $M_n$ such that:
\begin{equation}
|I_n| \leq M_n \quad \text{and} \quad \sum M_n < \infty,
\end{equation}
then the series converges uniformly. However, due to the Gibbs phenomenon and oscillatory nature of the Fourier series, the error remains nonzero near discontinuities, proving that uniform convergence fails.

\section{Absolute Convergence}
A Fourier series converges absolutely if:
\begin{equation}
\sum_{k=-\infty}^{\infty} |I_k| < \infty.
\end{equation}
Since $I_k \sim \frac{1}{k}$ for large $k$, the series behaves like the harmonic series:
\begin{equation}
\sum \frac{1}{k},
\end{equation}
which is known to diverge. Hence, the Fourier series does not converge absolutely.

\section{Conclusion}
The Fourier series representation of the RL circuit response exhibits different types of convergence properties:
\begin{itemize}
  \item \textbf{Pointwise Convergence}: The Fourier series converges pointwise everywhere except at discontinuities, where it converges to the average of the left and right limits due to the Gibbs phenomenon.
  \item \textbf{Uniform Convergence}: The Fourier series does not converge uniformly because the Gibbs phenomenon introduces a persistent oscillatory error near discontinuities.
  \item \textbf{Absolute Convergence}: The Fourier series does not converge absolutely since its terms decay as $\frac{1}{k}$, leading to divergence of the harmonic series.
\end{itemize}
These results indicate that while the Fourier series accurately represents the RL circuit response in a pointwise sense, its lack of uniform and absolute convergence affects practical considerations such as truncation error and approximation quality near discontinuities.
