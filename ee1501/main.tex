\documentclass{article}
\usepackage{amsmath}
\usepackage{listings}
\usepackage{xcolor}
\usepackage{karnaugh-map}
\definecolor{codegray}{gray}{0.95}

\lstset{
  backgroundcolor=\color{codegray},
  basicstyle=\ttfamily\footnotesize,
  frame=single,
  breaklines=true,
  language=Verilog
}

\begin{document}

\section*{Problem 1 : Priority Encoder Module}

The following Verilog module implements a 4-to-2 \textbf{priority encoder} with an additional \texttt{valid} output signal. It takes a 4-bit input vector \texttt{in[3:0]} and produces a 2-bit output \texttt{out[1:0]} representing the highest-priority asserted input (with bit 3 having the highest priority and bit 0 the lowest).

\begin{itemize}
	\item The \texttt{always @(*)} block ensures combinational logic.
	\item The \texttt{casez} statement allows for "don't care" conditions (using `z`) to simplify pattern matching.
	\item If one or more input bits are high, the encoder outputs the binary index of the highest-priority high bit and sets \texttt{valid = 1}.
	\item If no input bits are high, \texttt{valid} is set to 0 and \texttt{out} is set to 0.
\end{itemize}

\noindent Here's the Verilog code:

\begin{lstlisting}
module priorityEncoder(input [3:0] in, output reg [1:0] out, output reg valid);
  always @ (*) begin
    casez (in)
      4'b1zzz : begin valid = 1'b1; out = 2'd3; end
      4'b01zz : begin valid = 1'b1; out = 2'd2; end
      4'b001z : begin valid = 1'b1; out = 2'd1; end
      4'b0001 : begin valid = 1'b1; out = 2'd0; end
      default : begin
        valid = 1'b0;
        out = 2'd0;
      end
    endcase
  end
endmodule
\end{lstlisting}

\section*{Problem 2 : 4 bit Mod 16 synchronous Up Counter}

The \texttt{synchronousUpCounter} module implements a 4-bit synchronous up counter with asynchronous reset and enable control. Two versions of the counter are shown:

\subsection*{1. Implementation using T Flip-Flop K-Map Logic}
\begin{itemize}
	\item This version mimics a T flip-flop–based design derived from Karnaugh map simplifications.
	\item On every rising edge of the clock:
	      \begin{itemize}
		      \item If \texttt{reset} is high, the counter resets to 0.
		      \item If \texttt{enable} is high, the next state of each bit is determined by toggling behavior:
		            \begin{align*}
			            \texttt{count[0]} & \leftarrow \sim \texttt{count[0]}                                                                 \\
			            \texttt{count[1]} & \leftarrow \texttt{count[1]} \oplus \texttt{count[0]}                                             \\
			            \texttt{count[2]} & \leftarrow \texttt{count[2]} \oplus (\texttt{count[0]} \& \texttt{count[1]})                      \\
			            \texttt{count[3]} & \leftarrow \texttt{count[3]} \oplus (\texttt{count[0]} \& \texttt{count[1]} \& \texttt{count[2]})
		            \end{align*}
	      \end{itemize}
	\item This design mirrors the behavior of T flip-flops where toggling occurs when input T = 1.
\end{itemize}
\begin{center}
	\begin{tabular}{|c|c|c|c|c|c|c|c|c|c|c|c|}
		\hline
		\multicolumn{4}{|c|}{Present State} & \multicolumn{4}{c|}{Transition} & \multicolumn{4}{c|}{Next State}                                                                             \\
		\hline
		$Q_3$                               & $Q_2$                           & $Q_1$                           & $Q_0$ & $T_3$ & $T_2$ & $T_1$ & $T_0$ & $Q_3'$ & $Q_2'$ & $Q_1'$ & $Q_0'$ \\
		\hline
		0                                   & 0                               & 0                               & 0     & 0     & 0     & 0     & 1     & 0      & 0      & 0      & 1      \\
		0                                   & 0                               & 0                               & 1     & 0     & 0     & 1     & 1     & 0      & 0      & 1      & 0      \\
		0                                   & 0                               & 1                               & 0     & 0     & 0     & 0     & 1     & 0      & 0      & 1      & 1      \\
		0                                   & 0                               & 1                               & 1     & 0     & 1     & 1     & 1     & 0      & 1      & 0      & 0      \\
		0                                   & 1                               & 0                               & 0     & 0     & 0     & 0     & 1     & 0      & 1      & 0      & 1      \\
		0                                   & 1                               & 0                               & 1     & 0     & 0     & 1     & 1     & 0      & 1      & 1      & 0      \\
		0                                   & 1                               & 1                               & 0     & 0     & 0     & 0     & 1     & 0      & 1      & 1      & 1      \\
		0                                   & 1                               & 1                               & 1     & 1     & 1     & 1     & 1     & 1      & 0      & 0      & 0      \\
		1                                   & 0                               & 0                               & 0     & 0     & 0     & 0     & 1     & 1      & 0      & 0      & 1      \\
		1                                   & 0                               & 0                               & 1     & 0     & 0     & 1     & 1     & 1      & 0      & 1      & 0      \\
		1                                   & 0                               & 1                               & 0     & 0     & 0     & 0     & 1     & 1      & 0      & 1      & 1      \\
		1                                   & 0                               & 1                               & 1     & 0     & 1     & 1     & 1     & 1      & 1      & 0      & 0      \\
		1                                   & 1                               & 0                               & 0     & 0     & 0     & 0     & 1     & 1      & 1      & 0      & 1      \\
		1                                   & 1                               & 0                               & 1     & 0     & 0     & 1     & 1     & 1      & 1      & 1      & 0      \\
		1                                   & 1                               & 1                               & 0     & 0     & 0     & 0     & 1     & 1      & 1      & 1      & 1      \\
		1                                   & 1                               & 1                               & 1     & 1     & 1     & 1     & 1     & 0      & 0      & 0      & 0      \\
		\hline
	\end{tabular}
\end{center}
\subsection*{Karnaugh Map for $T_3$}
\begin{karnaugh-map}[4][4][1][$Q_1Q_0$][$Q_3Q_2$]
\manualterms{0,0,0,0,0,0,0,1,0,0,0,0,0,0,0,1}
\implicant{7}{15}
\end{karnaugh-map}
$T_3 = Q_2Q_1Q_0$
\subsection*{Karnaugh Map for $T_2$}
\begin{karnaugh-map}[4][4][1][$Q_1Q_0$][$Q_3Q_2$]
\manualterms{0,0,0,1,0,0,0,1,0,0,0,1,0,0,0,1}
\implicant{3}{11}
\end{karnaugh-map}
$T_2 = Q_1Q_0$
\subsection*{Karnaugh Map for $T_1$}
\begin{karnaugh-map}[4][4][1][$Q_1Q_0$][$Q_3Q_2$]
\manualterms{0,1,0,1,0,1,0,1,0,1,0,1,0,1,0,1}
\implicant{1}{11}
\end{karnaugh-map}
$T_1 = Q_0$

\subsection*{Karnaugh Map for $T_0$}
\begin{karnaugh-map}[4][4][1][$Q_1Q_0$][$Q_3Q_2$]
\manualterms{1,1,1,1,1,1,1,1,1,1,1,1,1,1,1,1}
\implicant{0}{10}
\end{karnaugh-map}
$T_0 = 1$
\newline
\noindent Code snippet:

\begin{lstlisting}
module synchronousUpCounter(
  input clk,
  input reset,
  input enable,
  output reg [3:0] count
);
  always @ (posedge clk or posedge reset) begin
    if (reset == 1)
      out <= 4'd0;
    else if (enable == 1) begin
      count[0] <= ~count[0];
      count[1] <= count[1] ^ count[0];
      count[2] <= count[2] ^ (out[0] & count[1]);
      count[3] <= count[3] ^ (count[0] & count[1] & count[2]);
    end
  end
endmodule
\end{lstlisting}

\subsection*{2. Implementation using Adder Logic}
\begin{itemize}
	\item This alternative version uses a standard binary incrementer (adder logic).
	\item On the rising edge of the clock:
	      \begin{itemize}
		      \item If \texttt{reset} is high, the counter resets to 0.
		      \item If \texttt{enable} is high, the counter is incremented by 1.
	      \end{itemize}
	\item This approach is more compact and commonly used in synthesizable designs.
\end{itemize}

\noindent Code snippet:

\begin{lstlisting}
/* implemented using adder logic */
module synchronousUpCounter (
  input clk, 
  input reset, 
  input enable, 
  output reg [3:0] out
);
always @ (posedge clk or posedge reset) begin
  if (reset == 1) out <= 4'd0;
  else if (enable == 1) begin
    out <= out + 1'b1;
  end
end
endmodule
\end{lstlisting}

\section*{Problem 3: Even Parity Generator}

The following Verilog module computes the \textbf{parity bit} of an 8-bit input vector.

\begin{itemize}
	\item The module takes an 8-bit input vector \texttt{data[7:0]}.
	\item It produces a single-bit output \texttt{parity}, which is the result of a bitwise XOR (exclusive OR) operation on all bits of \texttt{data}.
	\item The reduction operator \texttt{\^{}} (XOR reduction) is used to compute the parity efficiently:
	      \[
		      \texttt{parity} = data[0] \oplus data[1] \oplus \cdots \oplus data[7]
	      \]
	\item If the number of 1s in \texttt{data} is odd, the output is 1 (odd parity). If even, the output is 0.
\end{itemize}

\noindent Here is the Verilog code:

\begin{lstlisting}
module parity(input [7:0] data, output parity);
  assign parity = ^ data;
endmodule
\end{lstlisting}

\end{document}

\end{document}

